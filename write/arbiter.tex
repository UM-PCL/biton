\documentclass{article}
\usepackage{interval}
\usepackage[inline]{enumitem}
\usepackage{stmaryrd}
\usepackage{amsmath}
\usepackage{graphicx}
\newcommand{\inhib}{\relbar\mapsfromchar}

\begin{document}
\subsection{Temporal arbiter implementation}

Given priority scores from all the candidate qubits, an
arbiter circuit is needed to select the ones with the
highest priority, in order for their syndromes to be routed
to the allocated communication infrastructure for decoding.

The desired functionality is in essence a top-$k$ argmax,
which will give the indexes of the $k$ logical qubits that
produce the highest scores and thus are in most urgent need
of complex correction. This would be relatively
straightforward to implement the traditional way with
arithmetic comparators and index multiplexers. However, this
implementation carries a significant area cost for the
limited budget allowed by the integration density of current
superconducting fabrication processes.

To reduce the area and power requirements, as well as limit
the latency, we instead use a race logic based design for
the arbiter. First, the priority scores of each candidate
qubit are translated into temporal signals. A priority score
of $x \in \interval{0}{5}$ gets encoded into a pulse
arriving at time $T_x = x {\Delta}_t$. The encoded scores are
then passed through a temporal bitonic sorter. Such a
circuit is very well suited for temporal implementation, as
only two functional cells are needed to sort two signals,
leading to hardware efficiency impossible by other means.
Because we need the top-$k$ argmax of the signals, not just
their max, a sorter by itself is insufficient. Instead,
the signals passing through the bitonic sorter are also used
to control a small routing circuitry, which will lead select
signals $return_i, i \in \interval{0}{k-1}$ for each of the
top $k$ maximum outputs of the sorter through the same path
the corresponding data signals took to reach that output.
This ends in a pulse being emitted at port $sel_i, i \in
\interval{0}{n-1}$, for exactly those qubits whose priority
$x_i$ is in the top-$k$ largest. This approach avoids
complicated index multiplexing and decoding circuitry and
produces a directly usable multi-hot qubit selection.

The basic element of the temporal arbiter is the comparator
module, shown in fig$?$. Let $T(x)$ denote the time a pulse
arrives at port or wire $x$. No more than one pulse will
reach any port during an execution of our temporal circuit,
so we only need to represent the time for that pulse. In
cases where no pulse reaches $x$ during an execution, we
assign $T(x) = +\infty$.

Two input temporal signals at ports $x_a$ and $x_b$ are
sorted, such that the first output $x_{max}$ will re-emit
the larger of the two encoded priorities, meaning the one
for which a pulse arrived the latest, such that $T(x_{max})
= \max{T(x_a), T(x_b)} + \tau_c$, where $\tau_x$ a constant
delay of the sorting. The second $x_{min}$ emits the smaller
of $x_a, x_b$. In cases of ties the outputs are equivalent.

For the backward pass, the comparator module routes the
select signal $return_{max}$ to the port $sel_{a}$ if the
$x_a$ arrived after $x_b$ and to port $sel_{b}$ if it
arrived before it or at the same time. A pulse appearing at
$return_{min}$ will be routed the opposite way. We consider
the encoding for the backward pass to be simply binary
instead of temporal, so we don't care about the arrival
times of $return$ signals as long as they occur after the
inputs $x_a, x_b$. 

This gives us $sel_a = \begin{cases} return_{max} &
T(x_a)>T(x_b) \\ return_{min} & otherwise \end{cases}$.
$sel_b$ works similarly. By first passing encoded priorities
$x_a$ and $x_b$ through the comparator and then sending a
pulse in $return_{max}$, the $sel$ outputs will encode the
argmax of the two priorities. 

To build the complete top-$k$ module, we can combine these
comparator modules in an arrangement of a bitonic sorter,
with $x_{max}$ and $x_{min}$ ports connected to the input
$x_a$ or $x_b$ ports of the following comparators and $sel$
ports connected to the $return$ ports of the preceding
comparators in the same fashion. First we pass pulses at at
times encoding the priority scores of the $n$ logical qubits
to the $n$ input ports of the sorter circuit. After the
signals have propagated through the sorter, pulses are
issued at the $return$ ports corresponding to the top $k$
largest outputs of the bitonic sorter. These pulses will
follow the path the perspective priority signal took through
the comparators to get there, thus only the $k$ $x$ ports at
the inputs of the sorter will receive pulses at their
related $sel$ ports. This selection bitvector can then be
used directly to route the syndromes of the correct qubits
to the communication module for external decoding.

However, we can see that a large portion of the sorter is
dedicating to sorting signals that do not belong to the top
$k$, which is unnecessary. To improve on hardware
efficiency, we sparsify the bitonic sorter. First, we split
the $n$ inputs in groups of size $k$ and pass each group
through a full bitonic sorter of size $k$, implemented as
described above. We can then reduce the number of signals by
half by selecting only the $k$ largest outputs from each
pair of $k$-sized sorters. This is done the same way that
selecting the top half of values is done in bitonic sorting,
by applying a comparator to the $i$-th output of the first
sorter and the $(k-i)$-th one of the second sorter. We then
pass the $\frac{n}{2}$ signals through a similar sort and
reduce process. We repeat this until only $k$ signals are
left, from which the $return$ signals will originate. This
reduces the number of comparators from $\frac{n \log_2{n}
(\log_2{n}+1)}{4}$ to $(\frac{k \log_2{k} (\log_2{k}+1)}{4} +\frac{k}{2}) (2\frac{n}{k}-1)$, and the number
of comparators a signal goes through from $\frac{\log_2{n}
(\log_2{n}+1)}{2}$ to $\log_2{\frac{n}{k}}(\frac{\log_2{k}
(\log_2{k}+1)}{2} + 1)$, which minimizes
latency.

Since we only keep the $x_{max}$ output from the comparators
that implement the $2k$ to $k$ reduction between sort steps,
and only need to route the $return_{max}$ signal to a $sel$
port, we can use a smaller version of the comparator module
for these cases, saving additional resources.

\subsection{Comparator circuit}

To implement the forward pass of the comparator, we only
need two cells, one for the race logic primitive
\textit{First Arrival} (FA), which gives as the $\min$ of
two signals, and another one for \textit{Last Arrival} (LA),
which gives the $\max$. These can be implemented with a C
element for LA and an inverted C element for FA. Both of
theses cells are cheap area-wise and we have tuned them to
have similar delays, so differences in their propagation
delay do not cause the order of signals' arrival to change
for the length of the sorter network. Thanks to this,
synchronization to re-align the signals to their encoded
value is not needed for distances of encoded temporal values
$\Delta_t \sim 20ps$ we consider.

To implement the backrouting part of the comparator module,
we need \begin{enumerate*}[label=\emph{\alph*})] \item to
    detect whether $a$ or $b$ arrived first and \item to
      demultiplex the $return$ signals based on that.
\end{enumerate*} The first case corresponds to an inhibit
cell in temporal logic. $Inhibit(a,b)$ or $a \inhib b$
allows signal $b$ to pass to the output only if $a$ has not
previously arrived to block it. In superconducting circuits
this operation is typically instanciated via a clocked
inverter cell. However, because in our case we don't need
the temporal information of when $b$ arrived, but only need
the boolean information of whether it came second, we
instead use the cheaper Destructive Readout(DRO) cell, which
is the SFQ equivalent to the CMOS DFF. This choice also
makes it easier to reset the cell with an external signal
after the operation is done. Signal $a$ is mapped to the
clock port of the DRO and $b$ to the data input
port. If $a$ arrives before $b$, no pulse is stored in the
DRO and thus no pulse is produced at the output. If instead
the pulse at $b$ arrives first at the data port of the DRO,
it stores a pulse in the superconducting loop that is makes
it to the DRO output upon $a$'s arrival. In the case the two
signals $a$ and $b$ are close enough in time to be in the
undefined behavior region of the DRO, they encode the same
discrete temporal value, thus their comparison results in a
tie and either the presence or absence of a pulse at the DRO
output is a valid result for our purposes. This is important
for the correct operation of the system.

To demultiplex $return_{max}$, a Complementary output
Destructive Readout (DROC) cell is used. This cell produces
a pulse at it's first output upon clock arrival if it has
not received a pulse in the data input port since the last
clock pulse, and produces a pulse at the second output
otherwise. By plugging the result of the $a \inhib b$ DRO in
the data port of the DROC and the $return_{max}$ wire to
it's clock port, as well as connecting the first and second
DROC outputs to the comparator module's $sel_b$ and $sel_a$,
a pulse at $return_{max}$ gets routed to $sel_a$ if $a$ was
the last signal of the two and to $sel_b$ in the other case.
Because we might need to select both of the comparator's
operands for the top-$k$, we also need a second DROC with
it's outputs swapped to route $return_{min}$ to the select
signal of the input that came first. We use a splitter to
pass $a \inhib b$ to the data input of both DROCs and two
merger cells to combine their outputs for $sel_a$ and
$sel_b$. Four more splitters are used to send inputs $a$ and
$b$ to the FA, LA and DRO cells. The comparator module only
requires a handful of gates, significantly less than a
bit-parallel implementation would, which makes it ideal for
our resource constrained environment.

The arbiter circuit has been implemented in the PyLse
pulse-transfer simulator for various sizes. Measurements for
area and latency are shown in table x and fig y.

\begin{table}[]
\begin{tabular}{|l|l|l|l|}
\hline
k & n   & latency (ps) & area (JJ) \\ \hline
2 & 64  & 348          & 6696      \\ \hline
4 & 128 & 698          & 31248     \\ \hline
8 & 256 & 1284         & 116064    \\ \hline
\end{tabular}
\end{table}

\includegraphics*[width=0.5\textwidth]{diagrams/comparator.drawio.pdf}
\includegraphics*[width=0.5\textwidth]{diagrams/sort4.drawio.pdf}
\includegraphics*[width=0.5\textwidth]{diagrams/arbiterk4n16.drawio.pdf}
\includegraphics*[width=0.5\textwidth]{diagrams/circuit_comparator.drawio.pdf}

\end{document}
